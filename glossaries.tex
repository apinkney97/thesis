% This has all glossary definitions contained within it

\newglossaryentry{COSObject}{name={\textsc{cos} object},description={An object of the COS Object System, of type boolean, integer/real number, string, name, array, dictionary, stream, or the null object}}
\newglossaryentry{galley}{name={galley},description={A metal tray into which physical type is set}}

\newglossaryentry{measure}{name={measure},description={The standard length of a line of text}}
\newglossaryentry{kerning}{name={kerning},description={Small adjustments to the spacing between pairs of letters}}
\newglossaryentry{leading}{name={leading},description={The vertical distance from the baseline of one line of type to the next. In physical typesetting, this was altered by using thin strips of lead}}
\newglossaryentry{point}{name={point},description={A common unit of measure in typesetting, usually defined as $\frac{1}{72}$ of an inch}}
\newglossaryentry{justification}{name={justification},description={The process of adjusting spacing between (or within) words on a line to effect some change in the alignment of the text with respect to its left and right margins}}
\newglossaryentry{glyph}{name={glyph},description={The smallest typesettable item. Generally a glyph is a single character, but compounds of characters (such as ligatures and dipthongs) are also logical glyphs, though they may represent two separate characters}}
\newglossaryentry{raggedright}{name={ragged right},description={Text that is aligned to the left margin but not the right is said to be set \emph{flush left, ragged right}. Commonly also known as \emph{left justified} text. Similarly, \emph{right justified} text may also be referred to as being set \emph{ragged left, flush right}}}

\newacronym{pdf}{\textsc{pdf}}{Portable Document Format}
\newacronym{cpu}{\textsc{cpu}}{Central Processing Unit}
\newacronym{cog}{\textsc{cog}}{Component Object Graphic}
\newacronym{json}{\textsc{json}}{JavaScript Object Notation}
\newacronym{w3c}{\textsc{w3c}}{World Wide Web Consortium}
\newacronym{html}{\textsc{html}}{Hypertext Markup Language}
\newacronym{xml}{\textsc{xml}}{Extensible Markup Language}
