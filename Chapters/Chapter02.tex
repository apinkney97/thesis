\chapter{The Malleable Document}\label{ch:malleable}  % ``Aims \& objectives''?? Booooring
%The real aim of this project is to produce documents that are adaptable to multiple viewing
apertures, but do not require total reprocessing to do so.

%\smarginpar{Malleable documents!}

\marginpar{Can we have one physical representation which fits to all devices \emph{and} is
efficient? Should we perform some pre-processing to the document to tailor it to a particular
device?}

Amazon's Kindle Store model provides a good example of the need for multipurpose documents\ed one
purchase may be viewed on any number of platforms\ed the desktop app, the mobile phone app, the
tablet app, and on the Kindle itself. Amazon does not currently provide any special dispensation for
any of the platforms it supports\ed every one receives the same source document to display.
Consequently, a Kindle book is effectively a jack of all trades, and master of none: instead of
looking good on any one platform, the books look mediocre on \emph{every} platform. The real
questions here are as follows:

\begin{itemize}
 \item Is it possible to have one physical representation which typesets well on \emph{all}
platforms whilst remaining efficient enough not to cause excessive battery drain?
 \item Is it possible to identify invariant features common to many renderings to reduce the need
for processing on the display device?
 \item Should a more abstract, higher level of document be provided from which lower-level
device-specific derivatives can be produced?
\end{itemize}
  
The desire is to develop some document format which is neither as `solid' as a PDF nor as `liquid'
as HTML, but somewhere in between (perhaps termed a \emph{malleable document}). Something which is
mostly formed, but can be gently massaged to fit the display as necessary.


