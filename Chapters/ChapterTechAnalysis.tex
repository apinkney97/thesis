\chapter{Technical Analysis}\label{ch:techanalysis}


%\section{View-time Operations}

Chapter~\ref{ch:intro} (and in particular Section~\ref{sec:goodtypesetting}) provides an overview of some of the operations required when laying out a document. This chapter goes into a little more detail, and contrasts the operations required to view fixed and flowable documents against the operations required to view malleable documents.

\section{Fixed Document Formats}
Documents in a fixed format are rendered in a manner similar to the following:
{\singlespacing
\begin{lstlisting}
parse and tokenise the document layout instructions;
foreach (layout instruction) {
    interpret and execute the instruction;
}
paint the results to the screen;
\end{lstlisting}
}
Crucially, the layout instructions are declarative, and do not permit computation, which ensures that the document is always rendered identically.~\cite{Bagley2007}

\newpage
\section{Flowable Document Formats}

When a document in a flowable format is to be laid out, the process is as follows:
{\singlespacing
\begin{lstlisting}
parse the source to identify flowable blocks;
foreach (flowable block) {
    apply a line-breaking algorithm;
}
paint the results to the screen;
\end{lstlisting}
}

This process generates information similar to that contained in fixed-format documents, which is then used to drive the painting of contents to the screen.

The line breaking algorithm can be as simple as or complex as desired. In most cases, the line-breaking algorithm used by flowable documents is reasonably simple, and will take a first fit approach. This will be of the form:

{\singlespacing
\begin{lstlisting}
parse block to identify all possible breakpoints;
while (non-breakable items remain to be laid out) {
    place one item on the current line;
    if (there is not space for the next item) {
        adjust spacing between items to justify;
        move to a new line;
    }
}
\end{lstlisting}
}

As is noted in Section~\ref{sec:goodtypesetting}, first-fit algorithms do not generally result in well-typeset output. The simplest of these algorithms will not attempt to identify potential hyphenation points. More complex layout algorithms that search for ``optimal'' layouts usually require a considerable amount of backtracking.

\newpage
\section{Malleable Documents}

The layout algorithm for a malleable document is as follows. First, the penalties are calculated:

{\singlespacing
\begin{lstlisting}
foreach (included galley rendering) {
    compute penalty for the rendering at current page width;
}
select the rendering with the minimum penalty;
\end{lstlisting}
}
\noindent
Then, using the galley rendering that has the lowest penalty, the content is laid out onto the screen.
{\singlespacing
\begin{lstlisting}
foreach (paragraph-level item in selected galley rendering){
    foreach (line-level item in paragraph) {
        use precomputed data to place words;
    }
}
paint the results to the screen;
\end{lstlisting}
}
Crucially, a number of complex steps have been moved from view-time to compile-time (as they are for fixed-format documents) but without flowability being sacrificed:
\begin{itemize}
 \item The source is already parsed into a form optimised for layout
 \item The line-breaking algorithm has been entirely precomputed
\end{itemize}
One step has been added: each included galley rendering must be examined once before layout, in order to ascertain its ``penalty'' for use. The penalty is based entirely on the width of the page, and the \gls{measure} (width) of the galley rendering.%, and is described in more detail in Section~\ref{sec:layout}.

This penalty is calculated by taking the extra required horizontal whitespace (which can be envisaged as slack between columns) and weighting this to further penalise large numbers of columns. This weighting is achieved by multiplying by a smaller-than-linear function of the number of columns, such as a square root or logarithm.

Many low-power processors do not come with floating-point hardware as standard (for example the \textsc{arm} range of processors) which might suggest that these are poor choices of functions since they must be emulated using integer and bitwise operations only. This is not particularly important, for a number of reasons. Firstly, it has been shown previously~\cite{Lomont2003} that it is often possible to use mathematical analysis to find extremely good approximations for such functions. Secondly, the range of inputs to such a function would be limited to integers ranging from 1 to (in an extreme case) about 20, so the values could be pre-computed and stored in a lookup table. Thirdly, the penalty (and hence the root or logarithm) is calculated precisely once for each included galley rendering: in section~\ref{sec:inc-renderings} it is suggested that between three and seven galley renderings should be included in any one document. Given these facts, it is clear that there should be little impact from using such a function in the penalty calculation.

\vspace{2em}

Most importantly, since the line-breaking algorithm has been moved to compile-time, there is no longer any requirement to limit its complexity. In fact, should the need (or desire) arise, the text layout can be hand-tuned, or \emph{entirely hand-typeset}, with no consequences at view-time.


\section{Handling of Floats}

The grid-based layout system devised in Section~\ref{sec:gridlayout} works in a similar manner to a first-fit line-breaking algorithm, in that it places elements on the page in order, in the first place they will fit. In the case of this system, each element is a floatable figure or line of text. Elements that are the same size as a single grid cell, such as lines of text set in the main point size, can simply be placed in the first empty slot in the current column, or the first empty slot in the next column, should there be no empty spaces.  For the placement of elements that are larger than a single grid cell, there is some overhead required to step through the grid until a suitable position can be found. Once a position has been found, each grid cell that it overlaps must be marked as being reserved.

In the worst case, this algorithm does have a greater-than-linear time complexity. In practice, so long as the number of floats does not become excessive (which would cause the grid to be walked many times to search for suitably large gaps) the algorithm runs in linear time.

The placement of floats is subject to certain constraints: they must span integer multiples of columns, and can only be placed aligned to grid cells. This is very different to the model used for floats in \gls{html}, whereby floats may be positioned arbitrarily, and text flowed around them. Although a little more restrictive than \gls{html}, this system is capable of producing layouts that lend themselves to many types of document that are likely to be read on \ebook{} readers. The next chapter discusses this in more detail.

