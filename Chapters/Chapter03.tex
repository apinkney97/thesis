\chapter{A Simple Implemenation}\label{ch:paper}


\marginpar{Mention paper here \cite{Pinkney2011}\ed appendix. Give brief overview of current
system.}
The majority of the work to date on this project is described in the paper \emph{Reflowable
Documents Composed from Pre-rendered Atomic Components}\cite{Pinkney2011}, which was published and
presented at the 2011 ACM Symposium on Document Engineering. It is also attached to this document as
an appendix.

The paper outlines a very simple approach towards a malleable document format: the source document
is simply rendered into `galleys' of text (see \cite{Pinkney2011}, section 3) multiple times, using
a high-quality typesetting algorithm, with each rendering using a different galley width. The
document is thus now displayable on multiple screen sizes (each width rendering may additionally be
scaled by the display device to simulate point-size changes) without the need for the document to be
expensively retypeset. The `best fit' rendering is chosen at display time, based on the available
screen width and zoom level. The current implementation of this system is built around pre-existing
tools for the \emph{ditroff} typesetting system\cite{Kernighan1982} and Component Object Graphic
PDF\cite{Bagley2003}.

\section{Efficiency}
\marginpar{Computational\cite{Knuth1981} \& space.}
It is clear that computational efficiency must be maximised in order to minimise processor usage and
battery drain. In order to do this, it will almost certainly be necessary to sacrifice storage
efficiency; that is, filesizes are likely to be bigger. This is not necessarily a problem. Storage
is cheap and plentiful; currently it is far more practical and cost effective to increase storage
space than battery life. For this reason, we shall not concern ourselves too much with producing
documents with small filesizes, and shall instead concentrate on pre-computing as much of the
typesetting as possible, whilst retaining flexibility.

\section{Handling of edge cases}
\label{edgecases}
\marginpar{Corner cutting? Adjusting spacing\ldots word/letter spacing, glyph reshaping.
\cite{Bringhurst2008}}
The work described in \cite{Pinkney2011} and outlined above does not contain any special handling
for edge cases, although this will not be difficult to implement. In particular, the system is
currently restricted to displaying text only in the exact manner in which it was rendered. As an
example, let us assume that a particular point size has been selected (i.e.\ the zoom level has been
fixed); the best fit galley is then chosen for the page width. It may be the case that this
particular galley rendering, at this zoom level, on a page of this width results in large margins.
Perhaps the next galley size up is too wide to display. Currently, the penalty is simply to have to
put up with these large margins. However, there are some simple tweaks which can be used to reduce
how noticeable this effect is. As stated in section~\ref{goodtypesetting}, whilst performing
justification, it is occasionally appropriate to adjust letter spacing and glyph widths, in addition
to word spacing. Whilst readjusting these three 
parameters \emph{after} justification has taken place may reduce the typesetting quality (by making
any of them larger or smaller than allowed by the justification rules) in certain cases it may be
acceptable to do so, to fit the pre-rendered galley lines to a different width. Using the current
implementation, it is particularly simple to adjust the word spacing\ed in the PDFs produced, each
word is stored distinctly, and so can be repositioned as necessary. It is also very simple to
reshape the glyphs, to either stretch or shrink them laterally. All PDF drawing operators are
executed in accordance with the current \emph{transformation matrix}, which alters the coordinate
system\cite{Adobe2001}. Glyph reshaping can thus easily be performed on any word or line by altering
this before calling the requisite drawing operator. It is more difficult to alter the letter
spacing, particularly in the current implementation, as this will involve looking up font metrics,
and will be comparatively inefficient.


\section{Examples of these principles in practice}
Of particular interest and relevance to this thesis, is Amazon's recently announced Silk web browser
for their new Kindle Fire platform. Silk uses a similar paradigm to that outlined in
\cite{Pinkney2011} in order to serve web pages to the Kindle Fire, in a manner that requires less
processing by the device itself. Amazon makes use of its extensive computing power in its Cloud
service to prerender any webpage requested through Silk. The output is tailored specifically to the
device, enabling, for instance, layout of tables, or image scaling, to be performed on the Cloud
servers rather than locally on the device. Amazon has produced a five minute technical overview
video, available at \verb+http://www.youtube.com/watch?v=_u7F_56WhHk+


\section{Anticipated extensions}
\marginpar{Newspaper galley creation guidelines? look into\ldots}
Since the current work relies heavily on the setting of text into galleys, it would probably be
pertinent to research newspaper galley creation guidelines. These may be difficult to come by, since
newspapers have been electronically typeset for much of the last three decades. Nevertheless, the
information must be available somewhere, and could prove invaluable.

\marginpar{Device profiles\ldots what is appropriate? Image resolution? Page width renderings for
specific screen size?}
At some point, it will probably prove necessary to tailor the output document to the specific device
for which it is intended. Accordingly, profiles for these devices must be devised, which contain all
the necessary attributes to produce a document to fit. For example, image resolution must be
considered. Is there a necessity to use hi-resolution images for a document which is to be displayed
on a mobile phone screen? Perhaps not. Whilst a large image may be scaled down to fit, unless the
image is required to be zoomable, it seems sensible to precompute this scaling, and remove the need
for the device to perform it each time the image is displayed. Similarly, it may be sensible to
generate galleys only in sizes appropriate to the device. A mobile phone is clearly going to require
narrower renderings than a tablet, for example. Whilst each document \emph{could} include many
resolutions of each image, and a vast range of galley width renderings suitable for many devices, it
seems rather wasteful to do so when 
these tailored versions can easily be produced. It is envisaged that such a system would run on a
user's desktop or laptop PC, which would store the canonical, high-level representation of the
document, and that the apparent copying of that document to a display device would in fact create a
bespoke version of the document, specifically to fit the device, yet still with enough flexibility
to allow the font size to be changed within sensible limits.

\marginpar{Perhaps suggest using vector images where possible?}


\section{Conclusions}
\marginpar{Need to balance time/complexity against appearance. Comparable to eg Plass/Knuth? thus
need some way to quantify appearance\ed some form of metrics needed for this. Needs
investigation\ldots}
Computational efficiency is easily quantifiable, both on a theoretical level, and experimentally.
Typesetting quality, however, is harder to quantify. In order to show whether this system can
produce consistently better results than a standard eBook layout engine, some method must be derived
in order to compare them. This may be based on a point-scoring system, in a similar manner to
\TeX{}\cite{Knuth1984}, or perhaps by more qualitative means, such as human judgement. In either
case, both these factors must be balanced against one another, striving to reduce computational
efficiency whilst maintaining typesetting quality.


80