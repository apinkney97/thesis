\chapter{Foreword and Overview}
\setcounter{footnote}{-1}
This thesis is structured in a slightly unconventional manner:\footnote{but like all good computer scientists, we begin counting at zero\textsuperscript{$\star$}

 \vspace{0.8em}
\noindent \scriptsize{$\star$ except for the page numbering, because \LaTeX{} \emph{really} doesn't like having even numbers on right-facing pages\textsuperscript{$\dagger$}}

\vspace{0.8em} 
\noindent \tiny{$\dagger$ also the lack of a symbol for zero in Roman numerals rather spoils the joke}} its literature review is spread amongst the chapters, and sources are discussed within relevant areas of the text. Below is an outline of the thesis structure:

\vspace{1em}
\noindent Chapter~\ref{ch:intro} provides an overview of the present state of affairs of the electronic document world, with particular emphasis upon the technologies used in contemporary \ebook{} readers, and their benefits and drawbacks.

\vspace{1em}
\noindent Chapter~\ref{ch:malleable} takes a brief look at the history of movable type, and some of the techniques tradionally used in newspaper typesetting. Using some of these techniques as a basis, it then describes a novel paradigm for document representation that allows documents to fit a wide variety of screen sizes whilst maintaining high typographic quality. It then outlines a prototype implementation of a system to generate and display simple documents. Work in this chapter was published and presented at DocEng'11 in Mountain View, CA, USA.\hspace{0pt}\cite{Pinkney2011}

\vspace{1em}
\noindent Chapter~\ref{ch:floats} introduces a complete reimplementation of the system described in Chapter~\ref{ch:malleable} to enable its use on portable devices, and extends the devised model to include support for floating items such as figures and tables. Work in this chapter was published and presented at DocEng'13 in Florence, Italy.\hspace{0pt}\cite{Pinkney2013}

\vspace{1em}
\noindent Chapter~\ref{ch:bloat} addresses the issue of enlarged file sizes, and details various methods to keep file sizes to a minimum, where possible avoiding unnecessary increases computational complexity that would counter the work described in Chapters \ref{ch:malleable}~and~\ref{ch:floats}.

\vspace{1em}
\noindent Chapter~\ref{ch:analysis} provides technical and aesthetic analysis, focusing on the quantitative and qualitative aspects of the system as it runs at view-time, and includes the results of a user study of the system developed in the previous chapter.

\vspace{1em}
\noindent Finally, Chapter~\ref{ch:conclusions} evaluates the success of the project, and details areas of potential new research that have been encountered throughout the process.

