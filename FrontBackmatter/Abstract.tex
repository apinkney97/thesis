\cleardoublepage
%*******************************************************
% Abstract
%*******************************************************
%\renewcommand{\abstractname}{Abstract}
\pdfbookmark[1]{Abstract}{Abstract}
\begingroup
\let\clearpage\relax
\let\cleardoublepage\relax
\let\cleardoublepage\relax

\chapter*{Abstract}
Computer document processing often starts with an abstract, structural, representation before entering a processing pipeline that creates a desired layout and appearance. Unfortunately the whole system resembles the successive irreversible stages of generating assembler code using a compiler. This `one-way function' behaviour is most obvious with \textsc{pdf}, which is tied to a completely fixed appearance once a document has passed through a one-way `trapdoor', such as Adobe Distiller. Any attempt to reflow a document that has been processed through one of these one-way functions, or to view it at some other size, is either frustrating or simply impossible, without regenerating the document from a more abstract, higher-level representation.

Other formats, for example \textsc{html}, do maintain enough semantic structure to allow documents to be rendered to fit any page size. Since document renderers for \textsc{html}-like documents must run on-the-fly, they tend to produce results that are typographically inferior to those of document renderers whose runtime complexity is not constrained by time.

The limitations of these two opposing paradigms have not been cause for concern until fairly recently. In a world of smart phones, \ebook{} readers, laptops, tablets, smart watches, internet-connected televisions, and of course not forgetting the humble printed page, it is no longer safe to assume that a document will be viewed only in one fixed presentation. Current fixed formats do not lend themselves to having their presentational properties partially unpicked and re-engineered, nor are current `flowable' formats capable of reliably producing acceptably typeset documents. This thesis details work towards defining a format that stores documents that can be `repurposed' to fit any arbitrary screen size on any device whilst maintaining high typographic quality, and without the the need for total reprocessing.

\endgroup

\vfill