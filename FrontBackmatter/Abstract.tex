\cleardoublepage
%*******************************************************
% Abstract
%*******************************************************
%\renewcommand{\abstractname}{Abstract}
\pdfbookmark[1]{Abstract}{Abstract}
\begingroup
\let\clearpage\relax
\let\cleardoublepage\relax
\let\cleardoublepage\relax

\chapter*{Abstract}

Since the 1980s, two paradigms have dominated the representation of formatted electronic documents: flowable and fixed. Flowable formats, such as HTML, EPUB, or the output of a word processor, allow documents to scale to any arbitrary page size, but typographical compromises must be made since the layout is computed in real time, and is re-computed each time the document is displayed. Conversely, fixed formats such as SVG or PDF are afforded the potential for arbitrarily complex typography, but are constrained to the fixed layout that is set at the time of creation. With the recent surge in popularity of low-powered portable reading devices\ed{}from tablets to e-readers to mobile phones\ed{}there is an expectation that documents should scale to any size, maintain their high-quality typography, and not provide unnecessary strain on an already overloaded battery.

This thesis defines a novel paradigm for electronic document representation\ed{}the \emph{Malleable Document}\ed{}whereby documents are partially typeset at the time of creation, leaving enough flexibility that their content can be flowed to arbitrary page sizes with minimal computation. One tradeoff encountered is that of increased file size, and this is addressed with a bespoke, computationally-light compression scheme.

A sample implementation is presented that transforms documents from a source format into Malleable Document format, alongside a lightweight display engine that enables the documents to be viewed and resized on a wide range of devices. Reviews of the technical aspects and a user study to evaluate the quality of the system's rendering and layout show that the Malleable Document paradigm is a viable alternative to both fixed and flowable formats, and builds upon the best of both approaches.


\endgroup

\vfill