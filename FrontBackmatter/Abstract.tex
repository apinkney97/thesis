\cleardoublepage
%*******************************************************
% Abstract
%*******************************************************
%\renewcommand{\abstractname}{Abstract}
\pdfbookmark[1]{Abstract}{Abstract}
\begingroup
\let\clearpage\relax
\let\cleardoublepage\relax
\let\cleardoublepage\relax

\chapter*{Abstract}
Computer document processing often starts with an abstract, structural, representation before
entering a processing pipeline which creates a desired layout and appearance. Unfortunately the
whole system resembles the successive irreversible stages of generating assembler code using a
compiler. This `one-way function' behaviour is most obvious with \pdf{}, which is tied to a
completely
fixed appearance once a document has passed through a one-way `trapdoor', such as Adobe Distiller.
Other formats, for example \html{}, maintain enough semantic structure to allow documents to be
rendered to fit any page size, but even this breaks down under extreme circumstances. In essence,
any attempt to reflow a document, or view it at some other size, is either frustrating or simply
impossible, without regenerating the document from a more abstract, higher-level representation.

This limitation has not had much effect over the past 25 years, but it is becoming an ever more
pressing issue. In a world of smart phones, \ebook{} readers, netbooks, laptops, tablets,
30'' \textsc{lcd} screens, and of course not forgetting the humble printed page, it is no longer
safe to assume that a document will be viewed only in one fixed presentation. Current formats do not
lend themselves to having their presentational properties partially unpicked and re-engineered. This
thesis details work towards defining a document format suitable for `repurposing' with multiple
platforms and presentations in mind and with the intent of removing the need for total reprocessing.

\endgroup

\vfill